\documentclass[11pt,fleqn]{article}

\usepackage{color}
\usepackage{listings}

\begin{document}
\begin{center}
	{\Large \textbf{Lab 4 Report}}\\[6mm]
	\begin{tabular}{l}
        {\large Name: Ricky Fan}       \\
		{\large Email: fanh11@mcmaster.ca} \\
		{\large Student ID: 400248976}    \\
		{\large Date: 20/11/2022}
	\end{tabular}

\end{center}

\medskip

\subsection*{stats.c}
This C program simulates the dining-philosopher problem using 
a POSIX mutex lock and a POSIX condition variable. There are 
5 philosophers running on separate threads, each identified 
by a number from 0 to 4. Philosophers alternate states between 
thinking, being hungry (trying to pick up forks), and eating. 
To simulate the thinking and eating states, each thread will 
sleep for a random period between 1 to 3 seconds. When a 
philosopher is hungry, he must pick up 2 of his neighboring 
forks. If the forks are taken, the philosopher must 
wait until the forks are returned by others. Once a philosopher 
finishes eating and returns the forks, a cycle is completed. 
In this program, each philosopher is required to complete 
2 cycles to be considered finish dining.
The program can be run as \\

./dining-phil \\

\noindent
and will output the states of each philosopher, the availability 
status of each forks and the complete status of the philosophers.


\lstinputlisting[breaklines]{./dining-phil.c}
\end{document}